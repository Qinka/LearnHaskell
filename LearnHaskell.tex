%%%%%%%%%%%
%Haskell 讲义
%%%%%%%%%%%
\documentclass[UTF8]{ctexbook}

%package
\usepackage{color,titlesec,titletoc,listings}
\usepackage[colorlinks,linkcolor=blue,anchorcolor=blue,citecolor=red,bookmarksnumbered]{hyperref}

%程序清单的计数器
\newcounter{sourceCtr}
\setcounter{sourceCtr}{0}

%特定格式设计
%\lstset{emph={square}, emphstyle=\color{red},
%emph={[2]root,base},emphstyle={[2]\color{blue}}}

%lstlisting settings
%换行
\lstset{breaklines}
%背景色
\lstset{backgroundcolor=\color[rgb]{0.84,1.00,0.92}{}}
%格式
\lstset{basicstyle=\sffamily,keywordstyle=\bfseries,commentstyle=\rmfamily\itshape,escapechar='}
%非等宽
\lstset{flexiblecolumns}
%行号设置
\lstset{numbers=left,numberstyle=\tiny}
%针对汉字的逃逸字符
\lstset{escapechar=`}

\lstset{frame=trBL}
%标号
\lstset{label=sourceCtr}

%计数器格式设置
%\renewcommand\thechapter{\Roman{chapter}}

%格式设置
%part格式设置
\titleformat{\chapter}[rightmargin]{\bfseries\Huge}{\hbox{第 \thechapter 章}}{1em}{}[]
\titleformat{\section}[hang]{\bfseries\Large}{\thesection}{1em}{}[] 

%标题
\title{LearnHaskell}
%作者
\author{Qinka\\qinka@live.com}




%%宏
\def\sp{\phantom{.}}


\begin{document}
%封面
\maketitle
%版权与说明
%preface.tex
%序言
\newpage


\chapter*{前言}
Haskell 是一门十分不错的语言,具有纯函数式、惰性、静态类型等。笔者为了巩固Haskell的知识,并试图发觉新的内容,于是将所了解到的知识做一个笔记,也方便其他正在学习Haskell的读者去使用。

这个不知道算作什么的“文字产品”或许并未十分全面的,然而它还是会针对一些问题,给出解决。虽然笔者的资格什么的并不足以写这样的一份东西,然而还是去尝试了。有意将这个当作西安电子科技大学软件学院学生科协活动中Haskell的一个讲义了。

再次要感谢老爸在\LaTeX 排版上的帮助。
然而笔者的水平有限,语文水平更是,如果由各种各样的问题,那就恳请各位读者批评指正。

\footnote{该前言,也算作序了,是在开工之前编写的。}
\endinput 
%目录
\pdfbookmark[1]{text}{anchor}
\tableofcontents

%第一章
%chapter1.tex
%第一部分
%起步

\chapter{开始}
这一部分,解决的是初期环境的配置问题,以最基本的一些与Haskell语法语素无关的东西。

学习Haskell,你需要两样东西:编译器与编辑器。编辑器会影响编写代码的效率,而这正是为什么 Microsoft Visual Studio 是世界最好的IDE\footnote{集成开发环境(Integrated Development Environment)}的原因了。而编译器则选用GHC \footnote{Glasgow Haskell Compiler}。
还可以使用Haskell Platform。
\section{Haskell的编译器——GHC}
笔者现在使用的GHC的版本是 7.10.2,安装的方法因系统而异。笔者现在使用的是 Microsoft Windows 10 Pro 的系统。
\subsection{Windows}
GHC 的下载地址是 \url{http://www..haskell.org/ghc}。安装则需要将 压缩包解压后将 \verb"bin" 文件夹与 \verb"mingw/bin" 添加到 \verb"PATH" 环境变量中。

对于Haskell Platform下载地址 \url{https://www.haskell.org/platform}。直接点击安装包就好。
\subsection{Linux}
GHC 的下载地址GHC 的下载地址是 \url{http://www..haskell.org/ghc}。安装则需要将压缩包解压后。先下安装一些必要的软件,例如在Ubuntu下
\begin{lstlisting}[language=bash]
	sudo apt-get install libedit2 libedit-dev freeglut3-dev libglu1-mesa-dev libgmp3-dev
\end{lstlisting}

Haskell Platform 这需要去 \url{https://www.haskell.org/platform/windows.html#linux} 下载。
\subsection{OS X}
对于Haskell Platform 和 GHC 的安装方法请参考网上。

\section{效率利器-编辑器}
\begin{description}
  \item[Vim]\verb"vim" 编辑器是被广泛使用的编辑器。\verb"vim"自带 Haskell 的语法高亮。同时\verb"vim" 有 对 Haskell的交互——GHCi提供支持。详细信息访问\url{http://www.vim.org/scripts/script.php?script_id=2356}   
  \item[Emacs]\verb"emacs" 可以算是人气比较高的编辑器,使用又强大。emacs通体有Lisp编写,然而并没有什么卵用。。\\ 详细信息访问
  \url{https://wiki.haskell.org/Emacs}
  \item[Yi]\verb"Yi"是一个由Haskell编写而可由Haskell扩展。详细信息访问\url{https://wiki.haskell.org/Yi}
  \item[Sublime Text] \verb"Sublime Text" 是一个先进的编辑器。详细信息访问\url{https://wiki.haskell.org/IDEs#Sublime-Haskell}
  \item[Leksah] 也是一个由Haskell编写的编辑器。然而这实质上算作IDE,其附带Cabal文件的编辑器。笔者已经成功在Windows上与Ubuntu上编译。详细信息访问\url{http://www.leksah.org}
  \item[Atom] 这个编辑器是由Github,Inc 开发的开源编译器有多款支持Haskell的插件。详细信息访问\url{}
  \item[EclipseFP] 这个是一个基于Eclipse做的插件用于支援Eclipse编写Haskell。详细信息访问\url{}
\end{description}
其余可以使用的编辑器,参见\url{https://wiki.haskell.org/IDEs} 与 \url{https://wiki.haskell.org/Editors}
\endinput 


\end{document} 