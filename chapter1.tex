%chapter1.tex
%第一部分
%起步
%预编译内容

\makeatletter
\def\onePrecompile{\@ifundefined{markPrecompile}{\relax}{\endinput}}
\makeatother
\onePrecompile
%%%%%%%%%%%%%
\makeatletter


\def\aheadInfo{\@ifundefined{allBook}{\documentclass[UTF8]{ctexart}}{\relax}}
\def\documentBegin{\@ifundefined{allBook}{\begin{document}}{\relax}}
\def\documentEnd{\@ifundefined{allBook}{\end{document}}{\relax}}
\def\IfAllBool#1#2{\@ifundefined{allBook}{#2}{#1}}
\def\titleOrChaptter#1{\@ifundefined{allBook}{\title{Haskell讲义\\ [2ex] \begin{large} #1 \end{large}} \maketitle }{\chapter{#1}}}


\makeatother
%%%%%%%%%%%%%
\def\markPrecompile{\relax}
\endinput



\aheadInfo
%导言区

\makeatletter
\def\onePreamble{\@ifundefined{markPreamble}{\relax}{\endinput}}
\makeatother

\onePreamble

%package
\usepackage{color,titlesec,titletoc,listings}
\usepackage[colorlinks,linkcolor=blue,anchorcolor=blue,citecolor=red,bookmarksnumbered]{hyperref}

%程序清单的计数器
\newcounter{sourceCtr}
\setcounter{sourceCtr}{0}

%特定格式设计
%\lstset{emph={square}, emphstyle=\color{red},
%emph={[2]root,base},emphstyle={[2]\color{blue}}}

%lstlisting settings
%换行
\lstset{breaklines}
%背景色
\lstset{backgroundcolor=\color[rgb]{0.84,1.00,0.92}{}}
%格式
\lstset{basicstyle=\sffamily,keywordstyle=\bfseries,commentstyle=\rmfamily\itshape,escapechar='}
%非等宽
\lstset{flexiblecolumns}
%行号设置
\lstset{numbers=left,numberstyle=\tiny}
%针对汉字的逃逸字符
\lstset{escapechar=`}

\lstset{frame=trBL}
%标号
\lstset{label=sourceCtr}



%计数器格式设置
%\renewcommand\thechapter{\Roman{chapter}}

%格式设置
%part格式设置
\titleformat{\chapter}[rightmargin]{\bfseries\Huge}{\hbox{第 \thechapter 章}}{1em}{}[]
\titleformat{\section}[hang]{\bfseries\Large}{\thesection}{1em}{}[]

%作者
\author{Qinka\\qinka@live.com}




%%宏
\def\sp{\phantom{.}}

\def\markPreamble{}
\endinput 
\documentBegin
\titleOrChaptter{开始}
这一部分,解决的是初期环境的配置问题,以最基本的一些与Haskell语法语素无关的东西。

学习Haskell,你需要两样东西:编译器与编辑器。编辑器会影响编写代码的效率,而这正是为什么 Microsoft Visual Studio 是世界最好的IDE\footnote{集成开发环境(Integrated Development Environment)}的原因了。而编译器则选用GHC \footnote{Glasgow Haskell Compiler}。
还可以使用Haskell Platform。
\section{Haskell的编译器——GHC}
笔者现在使用的GHC的版本是 7.10.2,安装的方法因系统而异。笔者现在使用的是 Microsoft Windows 10 Pro 的系统。
\subsection{Windows}
GHC 的下载地址是 \url{http://www..haskell.org/ghc}。安装则需要将 压缩包解压后将 \verb"bin" 文件夹与 \verb"mingw/bin" 添加到 \verb"PATH" 环境变量中。

对于Haskell Platform下载地址 \url{https://www.haskell.org/platform}。直接点击安装包就好。
\subsection{Linux}
GHC 的下载地址GHC 的下载地址是 \url{http://www..haskell.org/ghc}。安装则需要将压缩包解压后。先下安装一些必要的软件,例如在Ubuntu下
\begin{lstlisting}[language=bash]
	sudo apt-get install libedit2 libedit-dev freeglut3-dev libglu1-mesa-dev libgmp3-dev
\end{lstlisting}
Haskell Platform 这需要去 \url{https://www.haskell.org/platform/windows.html#linux} 下载。
\subsection{OS X}
对于Haskell Platform 和 GHC 的安装方法请参考网上。
\section{效率利器-编辑器}
\begin{description}
  \item[Vim]\verb"vim" 编辑器是被广泛使用的编辑器。\verb"vim"自带 Haskell 的语法高亮。同时\verb"vim" 有 对 Haskell 的交互——GHCi 提供支持。详细信息访问: \url{http://www.vim.org/scripts/script.php?script_id=2356}。
  \item[Emacs] 可以算是人气比较高的编辑器,使用又强大。emacs通体有Lisp编写,然而并没有什么卵用。。\\ 详细信息访问
  \url{https://wiki.haskell.org/Emacs}。
  \item[Yi]是一个由Haskell编写而可由Haskell扩展。详细信息访问 \url{https://wiki.haskell.org/Yi}。
  \item[Sublime Text]  是一个先进的编辑器。详细信息访问\url{https://wiki.haskell.org/IDEs#Sublime-Haskell}
  \item[Leksah] 也是一个由Haskell编写的编辑器。然而这实质上算作IDE,其附带Cabal文件的编辑器。笔者已经成功在Windows 上与Ubuntu上编译。详细信息访问\url{http://www.leksah.org}。
  \item[Atom] 这个编辑器是由Github,Inc 开发的开源编译器有多款支持Haskell的插件。详细信息访问\url{https://wiki.haskell.org/IDEs#Atom}。
  \item[EclipseFP] 这个是一个基于Eclipse做的插件用于支援Eclipse编写Haskell。详细信息访问 \url{http://eclipsefp.github.io/}。。
\end{description}
其余可以使用的编辑器,参见\url{https://wiki.haskell.org/IDEs} 与 \url{https://wiki.haskell.org/Editors}。虽然上述的东西肯能不全,有限本质上算作插件而非编辑器。然而还希望有所帮助。
\section{软包的安装}
安装软件包的基本指令
\begin{lstlisting}[language=bash]
  $cabal update
  $cabal install [`报的名称`]
\end{lstlisting}

在终端中键入 \lstinline[language=bash]{cabal help}来获得帮助 \lstinline[language=bash]{cabal help command} 可以获得一些关于“command”的帮助。
\section{GHCi}
GHCi 是大部分时候“调戏”Haskell 的主战场下面是一些有用的命令。
\begin{itemize}
	\item :? 是获取帮助的命令
	\item :set prompt "ghci>" GHCi的默认提示符是 Prelude> 如果看的不顺眼,可以使用该命令替换。
	\item :t :type 这个命令是用来获取一个函数的类型绑定 \footnote{就如同 C 语言中的函数声明或者函数类型。}。
	\item :r :reload 当原文件被修改时,使用此命令重新导入。
	\item :l :load 导入指定文件夹下的文件。
	\item :edit 用于编辑当前已经导入的文件。
	\item :m :module 用于导入一个库 \footnote{也可以使用import 命令 只不过import 只能导入一个模块一次。}。导入Data.String 与 Data.Map 两个模块,再除去。
\begin{lstlisting}
  :m + Data.Map Data.String
  :m - Data.Map Data.String
\end{lstlisting}
	\item :q :quit 退出
\end{itemize}
\section{源文件、注释、模块与Hackage}
Haskell 的源文件的扩展名是 \verb".hs" 与 \verb".lhs" 后者是 Literate Haskell Script。后者通常会与 \TeX 或者 \LaTeX 一起出现。后者通过程序 lhs2tex 可以获得到tex代码 \footnote{请参阅 \url{http://www.andres-loeh.de/lhs2tex/}。}。
\begin{lstlisting}[language=Haskell,caption={lhs示例}]
> main :: IO ()
> main = putStrLn "Hello, Haskell!"
\end{lstlisting}



\begin{lstlisting}[language={[LaTeX]TeX},caption={lhs示例}]
\documentclass{article}
%include polycode.fmt
\begin{document}
This is the famous "Hello world" example,
written in Haskell:
\begin{code}
main  ::  IO ()
main  =   putStrLn "Hello, world!"
\end{code}
\end{document}
\end{lstlisting}

Haskell 的注释符号是 \verb"--" 与 \verb"{- note -}"。前者是单行注释,后置用于多行注释。\verb"{-# Options #-}" 这个则是有关于编译参数。
\begin{lstlisting}[language=Haskell,caption={引号替代}]
  {-# LANGUAGE QuasiQuotes #-}
  [hamlet|<p>This is quasi-quoted Hamlet.|]
\end{lstlisting}
Haskell 的模块——\verb"module"包含在各种Haskell的包之中。其中 Prelude 库是预加载库,在各种编译器、解释器与交互环境中是自动加载。 Haskell 中模块是包含在包中的。包主要是通过 Hackage 与 Cabal 安装。安装后模块再被“吊装”。

\subparagraph{Hackage} \footnote{\url{http://hackage.haskell.org}}
大部分 Haskell 的包都通过Hackage分发。他们的作者将打包好的包上传至Hackage上后以供人们下载。

\subparagraph{Cabal}\footnote{ \url{ http://www.haskell.org/cabal/}}
绝大部分的包都是通过Cabal 安装。它可以从hackage.haskell.org下载最新的包并安装。也可以从本地文件安装。

\subparagraph{Hoogle}\footnote{\url{http://hoogle.haskell.org}} 一个用于搜索 Haskell API的网站。
\subparagraph{Hayoo}\footnote{\url{http://hayoo.fh-wedel.de/}} 同样是一个 API搜索站。






\documentEnd
\endinput 